% Introduction can be seen as an expanded version of the summary. You can have roughly the same structure but with one or two pieces for each item in the summary. The following should be included:

% - Presentation of the area and topic of the work. This should come early and should capture the interest. This may include briefs on the background and possibly important definitions of concepts
% - You can briefly describe the intended target audience for the report, which ones have you written for?
% - A brief overview of previous work and their limitations
% - Presentation of the task including purpose and question
% - Description of how you attacked the task, method and why this is appropriate
% - Motivation: why is this task is interesting, what is the relevant issue(s), why is your approach good and why are the results important.
% - Description of the most important results and their limitations as well as what is new in your work
% - Overview of the report

% You can discuss the significance of the conclusions, but the introduction should only be briefly summarizing the results. No specialized terminology or mathematics should be included here.

% The introduction can be written as a funnel: area - sub-area - task - possible sub-task – purpose/aim. You then guide the reader towards a gradually more detailed and specific understanding of the task and purpose. At the end of the introduction, the reader and you should have a base of common understanding. The reader should understand the task, the scope of the work, the method and its most important contribution, i.e. what is new in your work.

% The other sections of the report may also need a brief introduction at the beginning, for the reader to understand the purpose of each section and its place in the report.


%    1 \section{section}
%    2 \subsection{subsection}
%    3 \subsubsection{subsubsection}
%    4 \paragraph{paragraph}

\section{Introduction}

Introduction to the document \& some text.

Example of how abbreviation system works. \gls{mac} is a layer 2 protocol, this layer 2 protocol with Ethernet has a  12 hex character address typically called \gls{mac}-address. This abbreviation system works as following the first time it sees a new \say{\textbackslash gls\{refrenceName\}} it types the full name and practices and ever other time it just uses the abbreviation.

Section levels is done via adding sub in font until level 4 were paragraph is used
\subsection{Level 2}
some text
\subsubsection{Level 3}
some text
\paragraph{Level 4}
\noindent some text

